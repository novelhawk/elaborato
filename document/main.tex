\documentclass{article}

\usepackage[T1]{fontenc} 
\usepackage[utf8]{inputenc}
\usepackage[italian]{babel} 
\usepackage[hidelinks]{hyperref} 
\usepackage{subfiles} 
\usepackage{tabularx} 
\usepackage{graphicx} 
\usepackage{pdflscape} 
\usepackage{booktabs}
\usepackage[square,numbers]{natbib}
\usepackage{environ}
\usepackage{caption}
\usepackage{tikz}
\usepackage{xcolor}

\graphicspath{{images}}
\bibliographystyle{unsrtnat}

\usetikzlibrary{shapes.misc, positioning}
\usetikzlibrary{calc}

\pgfdeclarelayer{bg}
\pgfsetlayers{bg,main}

\DeclareCaptionType{commandblock}[Access List][]

\hypersetup{
    colorlinks,
    linkcolor=LinkColor,
    citecolor=CiteColor,
    urlcolor=UrlColor
}

\pagecolor{white}
\color{black}
\colorlet{LinkColor}{black!10!red}
\colorlet{UrlColor}{black!20!magenta}
\colorlet{IndexLinkColor}{black}
\colorlet{CiteColor}{black!10!red}

\colorlet{CommandsHeader}{white!86!black}
\colorlet{CommandsBackground}{white!92!black}

\colorlet{CommandsHeaderText}{black}
\colorlet{CommandsBodyText}{black}

\input{components/command.tex}

\begin{document}

\subfile{content/0.title}

{
    \hypersetup{linkcolor=IndexLinkColor}
    \tableofcontents
}
\newpage

\setlength{\parskip}{0.6em}
\subfile{content/1.intro}
\subfile{content/2.analisi-requisiti}
\subfile{content/3.implementazione-servizio}
\subfile{content/4.analisi-funzionale}
\subfile{content/5.database}
\setlength{\parskip}{0em}
\subfile{content/6.database-modello-logico}
\setlength{\parskip}{0.6em}
\subfile{content/7.sicurezza-informatica}
\subfile{content/8.rete}

\nocite{*}

\bibliography{bibliography}

\end{document}

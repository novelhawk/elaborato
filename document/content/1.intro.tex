\section*{Prefazione}
A seguito della pandemia dovuta al COVID-19, l'esame di maturit\`a degli ultimi anni ha subito numerosi cambiamenti, tra questi compare un nuovo requisito per la prova di maturit\`a: l'\emph{elaborato}, un prodotto originale, coerente con la tematica assegnata dal Consiglio di Classe, realizzato da ogni studente con l'obbiettivo di accertare il livello di padronanza degli obiettivi e dei traguardi di competenza previsti dall'istituto~\cite{elaborato-intro}.

Ho scelto di produrre il mio elaborato sotto forma di documento scritto, e ho coinvolto principalmente discipline di indirizzo ma anche altre discipline del piano di studi.

\section{Introduzione}
Ogni elaborato ha una tematica diversa assegnata ad ogni studente dal Consiglio di Classe a seconda delle sue caratteristiche individuali e del suo livello di competenza. Ho riportato la tematica del mio elaborato di seguito:

\begin{center}
    \begin{minipage}{0.8\linewidth}
        Analisi e progettazione di un servizio come I-Beach, portale online che offre un
        servizio di prenotazione del posto in spiaggia libera o presso gli stabilimenti.
        Oltre agli aspetti puramente informatici devono essere esaminati anche gli aspetti
        della sicurezza dati durante il loro trasferimento nella rete e la loro
        memorizzazione. Progettare la rete LAN di uno stabilimento balneare in cui è presente
        anche un ristorante. 
    \end{minipage}
\end{center}

La consegna non va molto in dettaglio di proposito, le informazioni mancanti sono da interpretare liberamente da noi studenti.

Nel documento ipotizzer\`o di essere un'azienda di sistemi informatici e di essere stato commissionato questo progetto. Terr\`o in considerazione l'ambito del progetto e la sua portata in ogni scelta, cercando di essere il pi\`u realistico possibile.

La prima parte della consegna l'ho svolta nelle seguenti sezioni mentre ho svolto la progettazione della rete LAN nella sezione~\ref{sec:rete}.
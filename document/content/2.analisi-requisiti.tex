\section{Analisi dei requisiti}
Ci viene chiesto di realizzare un servizio sulle orme di \emph{I-Beach}, prima di tutto ho deciso di analizzare il sito web e informarmi sul suo funzionamento. L'applicazione permette di visualizzare le spiaggie disponibili nelle varie localit\`a e di prenotarle per una o pi\`u giornate con la possibilit\`a di pagare elettronicamente con PayPal e Carta di Credito. La prenotazione viene salvata nel profilo utente e viene inoltre inviata un email di conferma con tutte le informazioni sulla prenotazione e il QR Code associato. Le spiaggie che desiderano collaborare con il servizio possono richiedere informazioni tramite il sito web. Il sito web richiede una percentuale dell'incasso per ogni prenotazione, il restante viene inviato alle spiaggie a ricorrenza annua. Il sito web sembra includere anche altre funzionalit\`a tra cui la possibilit\`a di ordinare pasti remotamente e di prenotare i parcheggi. Queste tuttavia sembranno essere ancora in fase di sviluppo e fuori dall'ambito del progetto quindi svilupper\`o solo il servizio di prenotazione di spiaggie. Sono disponibili applicazioni sia per \emph{iOS} che per \emph{Android}. 

Dopo essermi informato su \emph{I-Beach} ho riorganizzato le idee e ho delineato i requisiti del progetto.

Il servizio dovr\`a essere online, dovr\`a permettere di prenotare le spiaggie e dovr\`a essere possibile pagare elettronicamente. Sar\`a necessario un database per salvare informazioni riguardato le spiaggie, le prenotazioni e gli utenti, inoltre, sar\`a necessario proteggere la comunicazione per garantire la sicurezza dei pagamenti.

Il servizio prevede anche informazioni per il pagamento annuale agli stabilimenti, questi dati sono considerati sensibili e perci\`o devono essere salvati separatamente in un'altro database preferibilmente in un'altra sede.

Il progetto richiede almeno un server e necessita di una buona protezione siccome conterr\`a dati personali. Considerato l'ambito del progetto consiglierei di optare per il Cloud Hosting siccome ridurrebbe considerevolmente i costi inziali e scala automaticamente con l'aumentare del numero di richieste tuttavia molte aziende vogliono essere sotto controllo dei dati dei propri utenti quindi, siccome l'hosting cloud delega questo controllo ad un'altra azienda, preferiscono ospitare il server localmente. L'impegno richiesto per garantire lo stesso livello di sicurezza delle aziende specializzate non \`e trascurabile a causa dell'elevata complessit\`a e dei costi non accessibili.

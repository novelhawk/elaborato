\section{Implementazione del Servizio}
Il Servizio pu\`o essere implementato in diverse forme: un sito web, un'applicazione cellulare, un'applicazione desktop o un'API\@, ognuna di queste pu\`o essere divisa successivamente in diverse sotto-categorie. Elenco di seguito le categorie che ho preso in considerazione per la realizzazione del Servizio.

\begin{itemize}
    \item \textbf{API}: Un'API \`e un servizio che permette di interagire direttamente con i dati, non ha un'interfaccia grafica ma attraverso richieste HTTP permette di richiedere, di aggiornare o di modificare le informazioni del Servizio. In questo progetto un'API pu\`o essere utilizzata come Layer di astrazione tra i Database e le eventuali interfaccie grafiche. Le API inoltre permettono di servire un sito web completamente statico.
    \item \textbf{Sito web}: I siti web sono i miglior candidati per questa tipologia di servizio. Permettono l'utilizzo del servizio da ogni piattaforma dotata di un browser e di una connessione a Internet. Il vantaggio principale dello sviluppo di un sito web risiede nella semplice integrazione in altre tipologie di applicativo: come spiegher\`o nei seguenti punti non un sito web pu\`o essere integrato all'interno di un'applicazione senza bisogno di scrivere codice aggiuntivo.
    \item \textbf{Sito web statico}: I siti web statici sono siti web che non utilizzano dei programmi server-side per generare delle risposte, invece rispondono sempre nello stesso modo a tutte le richieste. Il vantaggio dei siti web statici consiste nella possibilit\`a di salvare integralmente il sito in cache, permettendo il caricamento veloce delle pagine e anche la possibilit\`a di aprire la pagina senza connessione ad Internet.
    
    In questo caso le pagine devono contenere contenuto dinamico quindi sembrerebbe impossibile utilizzare questo approccio tuttavia \`e possibile servire un sito web statico che contiene codice Javascript responsabile di richiedere e inserire le informazioni nella pagina. In questo caso il codice viene eseguito client-side quindi \`e possibile salvare la pagina in cache e ogni volta che viene aperto il sito web invece di dover scaricare la pagina web, il codice Javascript si occuper\`a di aggiornare le informazioni e mostrarle a schermo.

    Questo approccio necessit\`a di API pubbliche accessibili da parte del client.
    \item \textbf{Sito web dinamico}: I siti web dinamici rispondono con delle pagine diverse ad ogni richiesta, le pagine vengono generate dal server a seconda dell'informazione richiesta. Questo genere di siti web non possono essere salvati in cache siccome la pagina contiene le informazioni richieste. I siti dinamici richiedono server performanti siccome il ritardo della creazione della pagina in risposta si ripercuote nell'utente finale che deve aspettare che la pagina completa venga creata prima di ricevere una risposta. \emph{I-Beach} \`e un sito web dinamico che utilizza la tecnica \textit{Applicazione cellulare browser} spiegata di seguito.
    \item \textbf{Applicazione Mobile}: Le applicazioni per cellulari hanno il maggior numero di utenti grazie alla comodit\`a e alla semplicit\`a d'utilizzo. Le applicazioni per cellulare rendono semplice la fidelizzare dei clienti siccome \`e probabile che l'applicazione venga utilizzata nuovamente dopo averla scaricata. Esistono numerose tecniche per lo sviluppo di applicazione per cellulare, ho tenuto in considerazione le seguenti.
    \item \textbf{Applicazione cellulare nativa}: Le applicazioni per cellulare possono essere scritte in vari metodi, in questa categoria includo le applicazione scritte nei linguaggi specifici per i singoli dispositivi, quindi Java per Android e Swift o Objective-C per iOS. In questo caso l'aspetto dell'applicazione sar\`a coerente con il cellulare utilizzato e raggiunger\`a le prestazioni pi\`u elevate. L'applicazione non sar\`a portabile tra i dispositivi ma sviluppare applicazioni native migliora l'esperienza utente siccome non ci saranno caricamenti lenti e applicazioni non risponsive. Spesso le applicazioni native devono comunicare con database esterni, in questo caso non \`e possibile utilizzare una comunicazione diretta con il database siccome le credenziali di accesso sarebbero estraibili dall'applicazione ma \`e necessario un layer intermedio che gestisca l'autenticazione, le API.
    \item \textbf{Applicazione cellulare browser}: Questo \`e l'approccio utilizzato da \emph{I-Beach}, l'applicazione non \`e altro che un browser che apre la pagina web del servizio simulando un'interfaccia che ricorda un'applicazione nativa. L'applicazione sar\`a pi\`u lenta alle applicazioni native ma per servizi come la prenotazione delle spiaggie dove la velocit\`a non \`e essenziale viene spesso utilizzato.
    \item \textbf{PWA}: Le Progressive Web App sono semplici siti web con l'aggiunta di un manifesto che contiene informazioni necessarie all'installazione del sito web come se fosse un'applicazione nativa. Funzionano nello stesso modo di \textit{Applicazione cellulare browser} ma lo sviluppo \`e pi\`u semplice. Permette l'installazione sia tramite i negozi dei dispositivi (Play Store e App Store) ma \`e anche possibile installarle con un click all'interno della pagina web.
    \item \textbf{Applicazione Desktop}: Tramite l'uso di \href{https://www.electronjs.org/}{Electron} \`e possibile integrare un sito web all'interno di un'applicazione desktop in pochi minuti con lo stesso principio di \textit{Applicazione cellulare browser} ma le applicazioni Desktop non vengono utilizzate per questo genere di servizio.
\end{itemize}

Per lo sviluppo del Servizio ritengo pi\`u opportuno l'utilizzo delle Progressive Web App. Questo permetter\`a di utilizzare il Servizio da tutte le piattaforme tramite browser ma permetter\`a anche di installarle come applicazioni nei dispositivi che supportano le PWA (tutti i dispositivi Google, Microsoft e Apple).

Per migliorare l'esperienza utente svilupperei il sito web staticamente, accompagnato da un servizio API pubblico. Questo approccio permette di salvare il sito integralmente nella memoria del dispositivo in cui \`e installato, permettendo l'apertura instantanea, anche quando la connessione Internet non \`e disponibile.

Il sito web utilizzer\`a richieste asincrone per ottenere le informazioni necessarie all'utilizzo del Servizio, rendendo il Servizio pi\`u fluido e responsivo. La presenza di API inoltre permette lo sviluppo di applicazione native se fosse ritenuto necessario.

\section{Analisi Funzionale}%
\label{sec:analisi_funzionale}

\subsection{Sito Web}%
\label{sub:sito_web}

Il sito web pu\`o essere implementato usando qualsiasi framework, sarebbe ideale utilizzare framework con rendering veloce per permettere un'apertura veloce delle applicazioni PWA. La dimensione delle librerie non \`e importante siccome possono essere salvate integralmente nella memoria del dispositivo. Ho deciso di utilizzare React perch\`e ho avuto esperienza personale con il framework e vanilla Javascript.

Sar\`a necessario aggiungere i manifesti iOS e Android per permettere l'installazione del sito web come applicazione Progressive Web App, inoltre sar\`a necessario creare il pacchetto d'installazione dell'applicazione se si vuole pubblicare all'interno degli store Google e Apple. L'installazione tramite bottone \`e possibile anche senza compilare il bundle applicativo.

Per quanto riguarda la sicurezza della trasmissione di dati \`e suffiente implementare il protocollo HTTPS nel server API, \`e quindi richiesto un certificato TLS/SSL che garantisca l'autenticit\`a e l'integrit\`a delle informazioni trasmesse. I certificati possono essere acquistati da enti certificatori (Certification Authority) ma un certificato semplice potrebbe essere gi\`a incluso con la registrazione del dominio o con l'hosting cloud. Il sito web non richiede protezione HTTPS siccome le informazioni trasmesse tra il sito e il visitatore sono statiche e non contengono informazioni personali tuttavia \`e conveniente installare un certificato semplice per non scoraggiare gli utenti all'inserimento di dati personali.

Ho scelto di utilizzare Cloudflare per proteggere il sito web, offre certificati SSL gratuiti ma \`e necessaria completa fiducia in Cloudflare siccome hanno accesso alle comunicazioni in chiaro (in dettaglio nella Sezione~\ref{sub:certificati}). Il certificato Cloudflare verr\`a usato solo per il sito web statico che non contiene informazioni personali quindi non costituisce un problema per noi.

\subsection{API}%
\label{sub:api}

La maggior parte delle web API di questo millennio hanno utilizzato la tecnologia REST (Representational state transfer), che associa ogni richiesta GET ad un'informazione e ogni richiesta POST, PUT, PATCH o DELETE ad un'azione, negli ultimi anni per\`o, le applicazioni web hanno iniziato ad introdurre numerose informazioni su singole pagine e la metodologia REST non scala con il numero di informazioni, infatti per ogni informazione l'utente deve inviare una nuova richiesta al server API\@. Per ovviare a questo problema \`e nato GraphQL\@.

GraphQL permette di richiedere solo le informazioni necessarie, riducendo le informazioni superflue ricevute in risposta dal server, inoltre permette di richiedere pi\`u informazioni insieme, ad esempio non \`e necessario ottenere prima le informazioni sulla spiaggia e poi i posti disponibili ma \`e possibile richiedere entrambe le informazioni nello stesso momento e anche per pi\`u spiaggie alla volta, riducendo il numero di richieste da centinaia ad una sola. % Bib: https://graphql.org/

Quindi la scelta della tipologia di API ricade su GraphQL ma le scelte non finiscono qui, infatti ora \`e necessario scegliere il linguaggio di programmazione da utilizzare per l'implementazione del servizio GraphQL\@. Sono disponibili librerie per ogni linguaggio ma ho deciso di utilizzare la libreria \href{https://github.com/graphql-rust/juniper}{Juniper} nel linguaggio Rust.

Rust \`e un linguaggio sviluppato da Mozilla con enfasi sulla sicurezza e sulla velocit\`a. Raggiunge velocit\`a equivalenti o superiori al linguaggio C ma le regole del linguaggio impediscono categoricamente la maggior parte delle vulnerabilit\`a ricorrenti nel linguaggio C~\cite{rust-website}. Lo sviluppo di applicazioni Rust non \`e semplice rispetto a linguaggi moderni come C\# e Javascript tuttavia la sua velocit\`a e la sua affidabilit\`a non \`e neanche confrontabile con questi linguaggi.

Rust riesce ad ottenere questa sicurezza tramite le restrizioni sull'utilizzo della memoria, infatti puntatori nulli o invalidi non sono permessi e non \`e possibile incorrere in \emph{race conditions}, inoltre non \`e presente un \emph{Garbage Collector} all'interno di Rust, le risorse sono acquisite durante l'inizializzazione e liberate quando escono fuori dallo \emph{scope}.

\subsection{Database}%
\label{sub:database}

Ho deciso di utilizzare il sistema di gestione di database \emph{PostgreSQL}. Il DBMS rispetta lo standard ISO ma offre numerose funzionalit\`a aggiuntive come il tipo \emph{text} per contenere stringhe di dimensione variabile. Il database deve essere installato in prossimit\`a del server API per ridurre il ritardo di comunicazione. Un secondo database \`e richiesto per salvare informazioni sensibili e il \emph{salt} utilizzato per crittografare le password degli utenti.

Le API si occuperanno di interfacciarsi con il Database. Ho scelto di utilizzare la libreria \href{https://diesel.rs/}{Diesel} per la comunicazione con il Database siccome \`e integrato molto bene con la libreria Juniper.

\subsection{Pagamento elettronico}%
\label{sub:pagamento_elettronico}

Per permettere ai clienti di prenotare spiaggie private \`e necessario implementare un servizio di e-commerce. Analizzando le varie possibilit\`a il servizio pi\`u adatto mi \`e sembrato essere \emph{PayPal Checkout}, le API sono semplici da integrare nel sito web e permette di pagare sia con PayPal che con Carte di Credito. Il servizio \emph{I-Beach} utilizza la stessa libreria.
